% Accessible APA 7 Beamer Presentation Template
% Author: Lanie Molinar Carmelo
% Template Version: 1.2.1 (2025-10-25)
% License: MIT License - https://opensource.org/licenses/MIT
%
% ACCESSIBILITY NOTES:
% - Uses semantic structure (sections, frames) not visual formatting
% - Presenter notes accessible via \note command
% - Color-independent design (no color-only cues)
% - Screen reader compatible with tagged PDF
%
% Usage: Fill in metadata, write content, and build with 'make presentation'
% NOTE: This template uses biblatex with the biber backend.
%       Compile with LuaLaTeX for best results and font support.

% \DocumentMetadata{
%   lang=en-US,
%   pdfversion=1.7,
%   pdfstandard=ua-1,
%   testphase={phase-III}
% }

\RequirePackage{expl3}
\documentclass[
  12pt,
  aspectratio=169,  % 16:9 widescreen (use 43 for 4:3)
  t,                % Top-align content (change to 'c' for center)
  ignorenonframetext % Ignore text outside frames for article mode
]{beamer}

% ===== THEME AND APPEARANCE =====
% Use a simple, accessible theme with clear structure
\usetheme{default}
\usecolortheme{default}

% Remove navigation symbols (not useful for screen readers)
\setbeamertemplate{navigation symbols}{}

% Use simple footline with page numbers
\setbeamertemplate{footline}[frame number]

% Simple itemize/enumerate (semantic, not decorative)
\setbeamertemplate{itemize items}[circle]
\setbeamertemplate{enumerate items}[default]

% Clear block structure
\setbeamertemplate{blocks}[default]

% ===== FONTS =====
% For maximum compatibility, use system fonts that are typically available
% Times New Roman for text (matches APA paper template)
% Liberation Sans (open-source Arial alternative) for slides
\usepackage{fontspec}
\setmainfont{Times New Roman}
% Use Liberation Sans if available, fallback to Latin Modern Sans
\IfFontExistsTF{Liberation Sans}{
  \setsansfont{Liberation Sans}
}{
  \setsansfont{Latin Modern Sans}  % Fallback for Beamer slide text
}
\usepackage{unicode-math}
\setmathfont{Latin Modern Math}

% Keep fonts readable
\setbeamerfont{title}{size=\Large,series=\bfseries}
\setbeamerfont{frametitle}{size=\large,series=\bfseries}
\setbeamerfont{framesubtitle}{size=\normalsize}

% ===== LANGUAGE AND CITATIONS =====
\usepackage[american]{babel}
\usepackage{csquotes}
\usepackage[style=apa,backend=biber,sortcites=true,sorting=nyt]{biblatex}
\DeclareLanguageMapping{american}{american-apa}
\addbibresource{references.bib}

% ===== ACCESSIBILITY AND METADATA =====
\usepackage{hyperref}
\hypersetup{
  pdftitle={Presentation Title},
  pdfauthor={Lanie Molinar},
  pdfsubject={Academic Presentation},
  pdfkeywords={APA, Beamer, Accessibility, Presentation},
  pdflang={en-US},
  colorlinks=true,
  linkcolor=blue,
  citecolor=blue,
  urlcolor=blue,
  bookmarksnumbered=true,
  pdfstartview=Fit
}

% ===== NOTES AND HANDOUT CONFIGURATION =====
% Conditional compilation for different output modes
\ifdefined\handoutmode%
  % Handout mode: 4 slides per page, no overlays
  \PassOptionsToClass{handout}{beamer}
  \AtBeginDocument{%
    \usepackage{pgfpages}
    \pgfpagesuselayout{4 on 1}[letterpaper,landscape,border shrink=5mm]
  }
\fi

\ifdefined\withnotes%
  % Notes mode: show notes below slides
  \setbeameroption{show notes}
\else
  % Default: no notes shown
  \setbeameroption{hide notes}
\fi

% ===== PRESENTATION METADATA =====
\title{Your Presentation Title}
\subtitle{Optional Subtitle}
\author{Lanie Molinar}
\institute{Colorado Christian University}
\date{\today}

% ===== DOCUMENT STRUCTURE =====
\begin{document}

% ===== TITLE FRAME =====
\begin{frame}
  \titlepage%
  \note{
    Welcome message and opening remarks.
    Introduce yourself and the topic.
    Duration: 30 seconds.
  }
\end{frame}

% ===== OUTLINE =====
\begin{frame}{Outline}
  \tableofcontents
  \note{
    Overview of presentation structure.
    Explain the flow of the talk.
    Duration: 1 minute.
  }
\end{frame}

% ===== SECTION 1: INTRODUCTION =====
\section{Introduction}

\begin{frame}{Introduction to the Topic}
  \framesubtitle{Setting the Context}

  \begin{itemize}
    \item First key point about the topic
    \item Second important concept
    \item Third foundational idea
  \end{itemize}

  \note{
    Explain the background and motivation for this topic.
    Connect to audience's existing knowledge.
    Mention why this matters.
  Duration: 2--3 minutes.
  }
\end{frame}

\begin{frame}{Research Question}

  \begin{block}{Central Question}
    What is the main research question or problem being addressed?
  \end{block}

  \pause%

  \begin{itemize}
    \item Why this question matters
    \item Current gaps in knowledge
    \item Expected contributions
  \end{itemize}

  \note{
    State the research question clearly.
    Explain its significance.
    Discuss what makes this question important or timely.
    Duration: 2 minutes.
  }
\end{frame}

% ===== SECTION 2: LITERATURE REVIEW =====
\section{Literature Review}

\begin{frame}{Previous Research}
  \framesubtitle{Key Studies in the Field}

  \begin{itemize}
    \item \textcite{smith2020} found that faith and reason are complementary
    \item \textcite{mounce2021} argues for biblical reliability
    \item \textcite{doe2021} examines theological perspectives
  \end{itemize}

  \pause%

  \begin{block}{Research Gap}
  Despite these contributions, there remains a need for\ldots
    Despite these contributions, there remains a need for\ldots
  \end{block}

  \note{
    Summarize relevant literature.
    Highlight key findings and debates.
    Explain how your work builds on or differs from prior research.
    Duration: 3 minutes.
  }
\end{frame}

% ===== SECTION 3: METHODOLOGY =====
\section{Methodology}

\begin{frame}{Research Approach}
  \framesubtitle{Methods and Design}

  \begin{enumerate}
    \item Data collection methods
    \item Analysis procedures
    \item Validation strategies
  \end{enumerate}

  \note{
    Explain your methodological approach.
    Justify your choices.
    Mention any limitations or challenges.
  Duration: 2--3 minutes.
  }
\end{frame}

% ===== SECTION 4: FINDINGS =====
\section{Findings}

\begin{frame}{Key Results}

  \begin{columns}[t]
    \begin{column}{0.5\textwidth}
      \textbf{Finding 1}
      \begin{itemize}
        \item Detail A
        \item Detail B
      \end{itemize}
    \end{column}

    \begin{column}{0.5\textwidth}
      \textbf{Finding 2}
      \begin{itemize}
        \item Detail C
        \item Detail D
      \end{itemize}
    \end{column}
  \end{columns}

  \note{
    Present main findings.
    Use clear, accessible language.
    Connect results to research question.
  Duration: 3--4 minutes.
  }
\end{frame}

\begin{frame}{Interpretation}

  \begin{block}{What This Means}
    Interpretation of the findings in context of existing literature
  \end{block}

  \begin{itemize}
    \item Theoretical implications
    \item Practical applications
    \item Unexpected outcomes
  \end{itemize}

  \note{
    Analyze and interpret results.
    Connect to broader themes.
    Acknowledge limitations.
  Duration: 2--3 minutes.
  }
\end{frame}

% ===== SECTION 5: CONCLUSION =====
\section{Conclusion}

\begin{frame}{Summary}

  \textbf{Main Takeaways:}
  \begin{enumerate}
    \item First key conclusion
    \item Second important finding
    \item Third significant insight
  \end{enumerate}

  \vspace{1em}

  \textbf{Future Directions:}
  \begin{itemize}
    \item Next steps for research
    \item Unanswered questions
  \end{itemize}

  \note{
    Summarize main points.
    Emphasize contributions.
    Suggest future research directions.
    Duration: 2 minutes.
  }
\end{frame}

% ===== REFERENCES =====
\begin{frame}[allowframebreaks]{References}
  \printbibliography[heading=none]
  \note{
    References are available for questions.
    Point out key sources if asked.
  }
\end{frame}

% ===== QUESTIONS =====
\begin{frame}{Questions?}

  \begin{center}
    \Large Thank you for your attention!

    \vspace{2em}

    \normalsize
    Contact: your.email@ccu.edu
  \end{center}

  \note{
    Be ready to answer questions.
    Have backup slides prepared if needed.
    Reiterate main message if appropriate.
  }
\end{frame}

% ===== APPENDIX (BACKUP SLIDES) =====
\appendix

\section{Appendix}

\begin{frame}[noframenumbering]{Additional Details}

  \begin{itemize}
    \item Supplementary information
    \item Detailed methodology notes
    \item Extended data or analysis
  \end{itemize}

  \note{
    Use these slides only if questions arise.
    Have specific examples ready.
  }
\end{frame}

\end{document}
